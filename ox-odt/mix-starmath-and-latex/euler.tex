% Created 2021-12-25 Sat 06:41
% Intended LaTeX compiler: pdflatex
\documentclass[11pt]{article}
\usepackage[utf8]{inputenc}
\usepackage[T1]{fontenc}
\usepackage{graphicx}
\usepackage{longtable}
\usepackage{wrapfig}
\usepackage{rotating}
\usepackage[normalem]{ulem}
\usepackage{amsmath}
\usepackage{amssymb}
\usepackage{capt-of}
\usepackage{hyperref}
\author{emacksnotes}
\date{\today}
\title{}
\hypersetup{
 pdfauthor={emacksnotes},
 pdftitle={},
 pdfkeywords={},
 pdfsubject={},
 pdfcreator={Emacs 29.0.50 (Org mode 9.5.1)}, 
 pdflang={English}}
\begin{document}

\tableofcontents

\begin{center}
\emph{(You are viewing this document in LATEX format)}
\end{center}

Hello  my friend!

How are you?

You asked me for some help with Euler's Identity. There is a very helpful article about this on
Wikipedia.  Here is an extract from the Wikipedia page.

Hope this helps.

Kind regards,

\emph{Emacksnotes}

\noindent\rule{\textwidth}{0.5pt}

\begin{center}
\uline{\textbf{On Euler's Identity}}
\end{center}

Euler's identity asserts that \(\mathrm{e}^{i\pi }\) is equal to −1.

The expression \(\mathrm{e}^{i\pi}\) is a special case of the
expression \(\mathrm{e}^{z}\), where \(z\) is any complex number.

In general \(\mathrm{e}^{z}\) is defined for complex \(z\) by
extending one of the definitions of the exponential function from real exponents to complex
exponents. One such definition is:

$$\mathrm{e}^{z}=\lim _{n\rightarrow \infty }\left(1+\frac{z}{n}\right)^{n}$$

Euler's identity therefore states that the limit, as n approaches infinity, of
\(\left(1+\frac{i\pi }{n}\right)^{n}\) is
equal to −1.

Euler's identity is a special case of Euler's formula, which states that for any real number x,

$$\mathrm{e}^{ix}\ =\ \cos (x)\ +\ i\ \sin (x)$$

where the inputs of the trigonometric functions sine and cosine are given in radians.

In particular, when \(x=\pi\),

$$\mathrm{e}^{i\pi }\ =\ \cos \pi \ +\ i\ \sin \pi $$

Since \(\cos \pi =-1\) and \(\sin \pi =0\)
it follows that

$$\mathrm{e}^{i\pi }=-1$$

which yields Euler's identity

$$\mathrm{e}^{i\pi }+1=0$$
\end{document}